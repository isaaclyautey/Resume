%%%%%%%%%%%%%%%%%%%%%%%%%%%%%%%%%%%%%%%%%
% Stylish Curriculum Vitae
% LaTeX Template
% Version 1.0 (18/7/12)
%
% This template has been downloaded from:
% http://www.LaTeXTemplates.com
%
% Original author:
% Stefano (http://stefano.italians.nl/)
%
% IMPORTANT: THIS TEMPLATE NEEDS TO BE COMPILED WITH XeLaTeX
%
% License:
% CC BY-NC-SA 3.0 (http://creativecommons.org/licenses/by-nc-sa/3.0/)
%
% The main font used in this template, Adobe Garamond Pro, does not
% come with Windows by default. You will need to download it in
% order to get an output as in the preview PDF. Otherwise, change this
% font to one that does come with Windows or comment out the font line
% to use the default LaTeX font.
%
%%%%%%%%%%%%%%%%%%%%%%%%%%%%%%%%%%%%%%%%%

\documentclass[a4paper, oneside, final, fontsize=9pt, usegeometry]{scrartcl} % Paper options using the scrartcl class

\usepackage{scrlayer-scrpage} % Provides headers and footers configuration
\usepackage{titlesec} % Allows creating custom \section's
\usepackage{marvosym} % Allows the use of symbols
\usepackage{tabularx,colortbl} % Advanced table configurations
\usepackage{fontspec} % Allows font customization
\usepackage{geometry}
\usepackage{enumitem}
\usepackage{colortbl}

\defaultfontfeatures{Mapping=tex-text}
\setmainfont{EB Garamond} % Main document font

\titleformat{\section}{\large\scshape\raggedright}{}{0em}{}[\titlerule] % Section formatting
\titlespacing*{\section}{0cm}{4pt}{4pt}

\pagestyle{scrheadings} % Print the headers and footers on all pages

\addtolength{\voffset}{0cm} % Adjust the vertical offset - less whitespace at the top of the page
\addtolength{\textheight}{3cm} % Adjust the text height - less whitespace at the bottom of the page
\geometry{margin={1cm}} % Adjust the margin size - horizontal whitespace

\newcommand{\gray}{\rowcolor[gray]{.90}} % Custom highlighting for the work experience and education sections

\newfontface\lserif{Liberation Serif}

\newcommand{\Csharp}{C{\lserif\#}}

\newcolumntype{Y}{>{\columncolor[gray]{0.9}}X}
\newcolumntype{Z}{>{\columncolor[gray]{0.9}}p{2cm}}

%----------------------------------------------------------------------------------------
% FOOTER SECTION
%----------------------------------------------------------------------------------------

\renewcommand{\headfont}{\normalfont\rmfamily} % Font settings for footer

\footheight=48pt % double check footer

\cofoot{
\addfontfeature{LetterSpace=20.0}\fontsize{8}{14}\selectfont % Letter spacing and font size
38 Hindsdale St {\large\textperiodcentered} Rochester {\large\textperiodcentered} New York {\large\textperiodcentered} 14620\\
\vspace{-3pt}
{\Large\Letter} isaac.lyautey@gmail.com \ {\Large\Telefon} (585) 773--5686
}

%----------------------------------------------------------------------------------------

\begin{document}

\setlist[itemize]{leftmargin=*}

\begin{center} % Center everything in the document

%----------------------------------------------------------------------------------------
% HEADER SECTION
%----------------------------------------------------------------------------------------

{\addfontfeature{LetterSpace=20.0}\fontsize{36}{36}\selectfont\scshape Isaac P. Lyautey}

%----------------------------------------------------------------------------------------
%	OBJECTIVE
%----------------------------------------------------------------------------------------

\section{Objective}
\begin{center}
    Obtain a result driven position in process/manufacturing engineering as part of a cross-functional team utlizing a proven business strategy of Total Productive Management in order to acheive a high degree of safety, quality, \& efficiency.
\end{center}

\vspace{-12pt}

%----------------------------------------------------------------------------------------
%	WORK EXPERIENCE
%----------------------------------------------------------------------------------------

\section{Work Experience}

% \vspace{-12pt}

\begin{tabularx}{0.97\linewidth}{XX}
    \vspace{-10pt}
    {\begin{tabularx}{0.97\linewidth}{>{\raggedleft\scshape}ZY}
        Period    & \textbf{June 2022 --- Present}                    \\
        Employer  & \textbf{Eastman Kodak}                    \\
        Location  & \textbf{Rochester, NY}                        \\
        Job Title & \textbf{Equipment Reliability Engineer} \\
    \end{tabularx}}
    \begin{itemize}\setlength\itemsep{0em}
        \item{Autonomous Maintenance \& Process Control}
        \begin{sloppypar}
            Implemented cleaning routines on battery coater reducing process waste. Lead operators educated on indicators poor process control and provided guidance through CAGs and SOPs to correct poor conditions. Conditions out of norm are escalated to maintenance staff or engineering as necessary. Turned firefighting into teaching events with documentation and guidance reducing reliance on support staff.
        \end{sloppypar}
        \item{Machine Spare Parts}
        \begin{sloppypar}
            Designed and implemented spare parts storage system with inventory tracking. Identified critical machine spares broken down by cost vs. potential downtime. Participated in price negotiations on capital orders.
        \end{sloppypar}
    \end{itemize} & \vspace{-10pt} \begin{itemize}\setlength\itemsep{0em}
        \item{Digitial Factory}
        \begin{sloppypar}
            Pi Vision displays have been created and brought to shop floor monitors which are actively used for process control and autonomous maintenance activities. This includes displays at control panels and larger overhead displays. Exposed process information through Pi Vision to operations and maintenance staff identifying key process parameters. 
        \end{sloppypar}
        \item{Data Analysis and Automation}
        \begin{sloppypar}
            Created automation program in Python to collect and merge process (Pi and Dr. Schenk) data with quality bench data. Demonstrated ability to analyze aggregate process data and drive improvement through statistical tests and quality metrics such as t-tests, ANOVA tests and CpK calculations. Worked with Quality Engineer and Technician to automate and streamline COA process reducing overhead. Fulfilled data requests of Quality Engineer and customer for: Defect size and type Pinhole count correlation to defect map Registration analysis and comparisons before and after process changes
        \end{sloppypar}
    \end{itemize}
\end{tabularx}

\vspace{-8pt}

\begin{tabularx}{0.97\linewidth}{XX}
    \hline
    \vspace{-6pt}
    {\begin{tabularx}{0.97\linewidth}{>{\raggedleft\scshape}ZY}
        Period    & \textbf{June 2021 --- June 2022}            \\
        Employer  & \textbf{Huhtamaki Inc.}                     \\
        Location  & \textbf{Fulton, NY}                         \\
        Job Title & \textbf{Continuous Improvement Specialist}  \\
    \end{tabularx}}
    \begin{itemize}\setlength\itemsep{0em}       
        \item{Six Sigma Green Belt}
        \begin{sloppypar}
            Lead 2 Six Sigma teams, the first tackling a printing process focusing on improving OEE by reducing scrap and non-value-added material use; The second team tackled productivity discrepancies between generations of forming equipment leading to large gains in sharing of ideas.
        \end{sloppypar}
        \item{Various Process Improvements}
        \begin{sloppypar}
            Improvements to process controls include oil tank monitoring systems, automation of clerical duties and leading a Process Control Systems rollout in all departments. PCS includes a combination of real-time machine performance and safety/quality/MEI reports displayed using Osisoft PI Vision.
        \end{sloppypar}
    \end{itemize} & \vspace{-10pt} 
    \begin{itemize}\setlength\itemsep{0em}
        \item{Total Productive Maintenance Leader}
        \begin{sloppypar}
            Lead a Kaizen/Lean event targeting a machine with high oil consumption which was heavily driven through teamwork between operations and maintenance staff. Through regular meetings of a cross-functional team multiple high yield opportunities were identified, trialed on the worst offending equipment and then dispursed to the entire fleet yielding COGS improvement.
        \end{sloppypar}
        \item{Data Analyst}
        \begin{sloppypar}
            Because of a strong background in data querying, aggregation and analytics learned through experience in software engineering and statistical research I was tasked with backfilling the Operations Analyst position while a replacement was sought for 4 months. Tasks included production reporting corrections by use of analytics, monthly MEI roundups and training of new-hire. 
        \end{sloppypar}
    \end{itemize}
\end{tabularx}

\vspace{-8pt}

\begin{tabularx}{0.97\linewidth}{ZY|ZY}
    \gray{}Period    & \textbf{January 2019 --- August 2019} & Period & \textbf{January 2020 --- August 2020}\\
    \gray{}Employer  & \textbf{Quest Global}                 & Employer & \textbf{Howmet Aerospace} \\
    \gray{}Location  & \textbf{Windsor Locks, Connecticut}   & Location & \textbf{Niles, Ohio}\\
    \gray{}Job Title & \textbf{Industrial Engineer Co-op}          & Job Title & \textbf{Process Engineer Co-op}\\
\end{tabularx}

\begin{tabularx}{0.97\linewidth}{X|X}
    \vspace{-10pt}
    \begin{itemize}\setlength\itemsep{0em}
        \item{Labor Variance \&{} Capacity}
        \begin{sloppypar}
            Collected the production demand, clock hours and part routings to map predicted vs actual labor times across all operations in all cells.
            Data was collected and compiled into a SQL database and through various manipulations produced a view for PowerBi interaction.
        \end{sloppypar}
        \item{Playbook, Task Scheduling \&{} Part Tracking}
        \begin{sloppypar}
            Facilitated factory-wide events to analyze the production-pacing process and find ways to improve productivity.
            Improvements included ergonomic adjustments, improved fixtures, layout adjustments and per-shift scheduling.
            Implemented an automated framework for part tracking and progression using SQL, \Csharp{} and VB.NET
        \end{sloppypar}
    \end{itemize} & \vspace{-10pt} \begin{itemize}\setlength\itemsep{0em}
        \item{In Process Checks \&{} Operator Training}
        \begin{sloppypar}
            Created and implemented Standard Work Procedure in previously uncontrolled process to reduce said variability.
            Replaced in-process engineering checks with SWP defining expectations of the process, common defect scenarios
            and defined escalation paths when tolerance is endangered.
        \end{sloppypar}
        \item {Automated Inspection Data Collection}
        \begin{sloppypar}
            Work with dimensional inspection operators to create a streamlined data entry interface which reduced
            input error and increased readability over the old system both on the operator's end and engineering's. This
            app incorporated WPF, EF, and SharePoint.
        \end{sloppypar}
    \end{itemize}
\end{tabularx}

%----------------------------------------------------------------------------------------
%	Projects
%----------------------------------------------------------------------------------------

% \section{Notable Projects}

% % MSD
% \begin{tabularx}{\linewidth}{>{\raggedleft\scshape}p{3cm}XX}
%     Inclusivity Retractable Game Net
%     & \vspace{-16pt} \begin{itemize}\setlength\itemsep{0em}
%         \item{Design a production ready handicapped assistive device in a structured team environment.}
%         \item{Interface with customer for use cases, design requirements and necessary standards to adhere to.}
%         \item{Rapid prototyping of CAD models through use of 3D printed plastics entailing unique dimensional tolerancing.}
%     \end{itemize}
% \end{tabularx}

%----------------------------------------------------------------------------------------
%	EDUCATION
%----------------------------------------------------------------------------------------

\vspace{-14pt}
\section{Education}

% RIT
\begin{tabularx}{0.97\linewidth}{>{\raggedleft\scshape}p{2cm}X>{\raggedleft\scshape}p{2cm}X}
    Degree & \textbf{Bachelor of Science in Mechanical Engineering} & School & \textbf{Rochester Institute of Technology}\\
    Period & \textbf{August 2018 --- May 2021} & GPA & \textbf{3.51}\\
\end{tabularx}

% RIT Classes
\begin{tabularx}{0.97\linewidth}{>{\raggedleft\scshape}p{2cm}X}
    Classes & Classical Controls \hfill CFD \hfill Fluid Mechanics I\textbackslash{}II \hfill Stochastic Processes \hfill Probability \&{} Statistics I\textbackslash{}II\\
\end{tabularx}

% % MCC
% \vspace{6pt}
% \begin{tabularx}{0.97\linewidth}{>{\raggedleft\scshape}p{2cm}XX}
%     \gray{}Period & \textbf{August 2015 --- May 2018} & \phantom \hfill\\
%     \gray{}Degree & \textbf{Associates in Engineering Science} & School \textbf{Monroe Community College} \hfill\\
% \end{tabularx}

%----------------------------------------------------------------------------------------
%	SKILLS
%----------------------------------------------------------------------------------------

\vspace{-6pt}

\section{Skills}
\begin{tabularx}{0.97\linewidth}{X|X}
    {\begin{tabularx}{0.97\linewidth}{>{\raggedleft\scshape}p{2cm}X}
        Prototyping & 3D Printing, CNC Programming, Arduino Microcontroller, RPi, Water Jet, Welding\\
        Tools & Excel, Matlab, PowerBi, SAP, SharePoint, PI, PI Vision, PI AF\\
        Languages & \Csharp{}, VB/A, Python, SQL, Java\\
    \end{tabularx}}&
    {\begin{tabularx}{0.97\linewidth}{>{\raggedleft\scshape}p{2cm}X}
        Manufacturing & Ultrasonic Non-Distructive Testing, Dimensional Inspection, In-line defect detection \& classification, Web Guidance, Conveyance\\
        CAD & Solidworks, AutoCAD, PTC Creo, Onshape\\
        Other & Linux, Git\\
    \end{tabularx}}
\end{tabularx}

%----------------------------------------------------------------------------------------

\end{center}
\end{document}
